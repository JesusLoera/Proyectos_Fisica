\section{El atractor de Lorentz}
\label{El atractor de Lorentz}

Uno de los sistemas más icónicos cuando se empieza el estudio de la teoría del caos es el de las 
\emph{ecuaciones de Lorenz} . 

\begin{definition}[Ecuaciones de Lorenz]
    \begin{equation}
        \dot{x} = \sigma (y-x) 
    \end{equation}
    \begin{equation}
        \dot{y} = x(\rho - z) - y 
    \end{equation}
    \begin{equation}
        \dot{z} = xy - \beta z
    \end{equation}
\end{definition}

El atractor de Lorenz es un concepto introducido por Edward Lorenz en 1963. Se trata de un sistema dinámico 
determinista tridimensional no lineal derivado de las ecuaciones simplificadas de rollos de convección que 
se producen en las ecuaciones dinámicas de la atmósfera terrestre.

Al parametro $\sigma$ se le suele denominar el número de Prandtl y $\rho$ se llama el número de Rayleigh.

