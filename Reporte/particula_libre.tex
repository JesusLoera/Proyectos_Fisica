\section{La partícula libre}
\label{La partícula libre}

En el contexto de la mecánica cuántica, las partículas cuánticas muestran características tanto de ondas como de 
partículas (clásicamente hablando). Es por eso que es necesario implementar un esquema matemático adecuado que englobe 
ambas características de forma simultánea y satisfactoria.\\
En la física clásica, una partícula está bien localizada en el espacio, dado que su posición y velocidad pueden ser 
calculadas simultáneamente con precisión arbitraria. Mientras que en la mecánica cuántica, una partícula se describe 
mediante una función de onda correspondiente a la onda de materia asociada con la partícula \emph{(Hipótesis de De Broglie)}. 
Dada la situación, una partícula que sea localizable en una cierta región del espacio puede ser descrita por una onda de materia 
cuya amplitud sea grande en esa región y cero fuera de ella. De esta manera, la onda de materia será localizada dentro 
de la región del espacio donde la partícula esté físicamente confinada.\\
Una función de onda localizable es llamada un \emph{paquete de ondas}. Un paquete de ondas consiste entonces de un grupo 
de ondas de ligeramente distintas longitudes de ondas, con fases y amplitudes tales que interfieran constructivamente 
dentro de una pequeña región del espacio y destructivamente fuera de ella.\\
El concepto de paquete de onda representa entonces una herramienta matemática unificadora que puede lidiar tanto con el 
comportamiento corpuscular de la naturaleza, así como con su comportamiento ondulatorio.