\newpage
\onecolumn

%Apendice 1 Transformada Discreta de Fourier
\section*{Apéndice A}
\label{ApendiceA}

\emph{Método en python con el algoritmo de la DFT}

\begin{python}
    def DFT(F, N):
    Im_F = F[0]
    Re_F = F[1]
    Im_G = []
    Re_G = []
    G = []
    Freq = []

    for k in range(N):
    im_g = 0
    re_g = 0
    for n in range(N):
    re_g = (
    re_g
    + np.cos((2 * np.pi * k * n) / (N)) * Re_F[n]
    + np.sin((2 * np.pi * k * n) / (N)) * Im_F[n]
    )
    im_g = (
    im_g
    + np.cos((2 * np.pi * k * n) / (N)) * Im_F[n]
    - np.sin((2 * np.pi * k * n) / (N)) * Re_F[n]
    )
    Im_G = np.append(Im_G, im_g)
    Re_G = np.append(Re_G, re_g)
    G = np.append(G, [re_g, im_g])
    Freq = np.append(Freq, k)

    return G, Re_G, Im_G, Freq
\end{python}



%Apendice 2 FFT
\section*{Apéndice B}
\label{ApendiceB}

\emph{Método en python con el algoritmo de la FFT}

\begin{python}
    def FFT(f):
    N = len(f)
    if N <= 1:
    return f

    # division
    even = FFT(f[0::2])
    odd = FFT(f[1::2])

    # store combination of results
    G = np.zeros(N).astype(np.complex64)

    # only required to compute for half the frequencies
    # since u+N/2 can be obtained from the symmetry property
    for u in range(N // 2):
    G[u] = even[u] + np.exp(-2j * np.pi * u / N) * odd[u]  # conquer
    G[u + N // 2] = even[u] - np.exp(-2j * np.pi * u / N) * odd[u]  # conquer

    return G
\end{python}

\newpage

%Apendice 3 Simulación Lorenz
\section*{Apéndice C}
\label{ApendiceC}

\emph{Código donde se simuló el atractor de Lorenz}

\begin{python}
    import numpy as np
    import matplotlib.pyplot as plt
    from scipy.integrate import odeint
    from Methods import FFT

    """
    El atractor de Lorentz
    """


    def Lorentz_ED(xyz, t, sigma, rho, beta):
    x, y, z = xyz
    return [sigma * (y - x), x * (rho - z) - y, x * y - beta * z]


    """
    Discretizamos la variable dependiente en N=2**m terminos,
    el cual es un requerimento de la Transformada Rapida de Fourier
    """
    m = 9
    N = 2 ** m


    """
    Condiciones iniciales y parametros del Atractor de Lorentz
    """

    # Asignamos valores a los parametros
    sigma, Rho, beta = 10, np.linspace(24, 25, 100), 8.0 / 3.0

    # Condicion inicial y valores de t sobre los que calcular
    xyz0 = [1.0, 1.0, 1.0]
    xi = 0.0
    xf = 25.0
    t = np.linspace(xi, xf, N)
    iter = 1

    for rho in Rho:

    # Resolvemos las ecuaciones
    xyz = odeint(Lorentz_ED, xyz0, t, args=(sigma, rho, beta))

    # Graficamos las soluciones
    from mpl_toolkits.mplot3d.axes3d import Axes3D

    fig, (ax1) = plt.subplots(1, 1, subplot_kw={"projection": "3d"})
    ax1.plot(xyz[:, 0], xyz[:, 1], xyz[:, 2], "r", alpha=0.5)
    ax1.set_xlabel("$x$", fontsize=16)
    ax1.set_ylabel("$y$", fontsize=16)
    ax1.set_zlabel("$z$", fontsize=16)
    plt.title("Diagrama de fase")
    plt.savefig("images_numpy_24_25/fase/fase" + str(iter) + ".png")
    plt.close()

    """
    Calculamos la FFT de X(t) de las ecns de Lorentz
    """
    G_Lorentz_X = FFT(xyz[:, 0])

    """
    Creamos un arreglo con los espectros de potencia de la FFT
    """
    S_fft = []
    for i in range(N):
    s = (1.0 / N) * abs(G_Lorentz_X)
    S_fft = np.append(S_fft, s)

    """
    Graficamos la funcion Fx_fft
    """
    plt.plot(t, xyz[:, 0])
    plt.title("X(t)")
    plt.savefig("images_numpy_24_25/eje_x/eje_x" + str(iter) + ".png")
    plt.close()

    """"
    Ahora Graficamos S vs frecuencia (Espectro)
    """
    Freq = np.arange(0, N // 2)
    plt.plot(Freq, S_fft[: N // 2])
    plt.title("Espectro de Frecuencias")
    plt.savefig("images_numpy_24_25/espectro/espectro" + str(iter) + ".png")
    plt.close()

    iter = iter + 1
\end{python}