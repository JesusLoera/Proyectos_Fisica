\newpage
\onecolumn

%Apendice 1 Transformada Discreta de Fourier
\section*{Apéndice A}
\label{ApendiceA}

\emph{Método en python con el algoritmo de la DFT}

\begin{python}
    def DFT(F, N):
    Im_F = F[0]
    Re_F = F[1]
    Im_G = []
    Re_G = []
    G = []
    Freq = []

    for k in range(N):
    im_g = 0
    re_g = 0
    for n in range(N):
    re_g = (
    re_g
    + np.cos((2 * np.pi * k * n) / (N)) * Re_F[n]
    + np.sin((2 * np.pi * k * n) / (N)) * Im_F[n]
    )
    im_g = (
    im_g
    + np.cos((2 * np.pi * k * n) / (N)) * Im_F[n]
    - np.sin((2 * np.pi * k * n) / (N)) * Re_F[n]
    )
    Im_G = np.append(Im_G, im_g)
    Re_G = np.append(Re_G, re_g)
    G = np.append(G, [re_g, im_g])
    Freq = np.append(Freq, k)

    return G, Re_G, Im_G, Freq
\end{python}



%Apendice 1 Transformada Discreta de Fourier
\section*{Apéndice B}
\label{ApendiceB}

\emph{Método en python con el algoritmo de la FFT}

\begin{python}
    def FFT(f):
    N = len(f)
    if N <= 1:
    return f

    # division
    even = FFT(f[0::2])
    odd = FFT(f[1::2])

    # store combination of results
    G = np.zeros(N).astype(np.complex64)

    # only required to compute for half the frequencies
    # since u+N/2 can be obtained from the symmetry property
    for u in range(N // 2):
    G[u] = even[u] + np.exp(-2j * np.pi * u / N) * odd[u]  # conquer
    G[u + N // 2] = even[u] - np.exp(-2j * np.pi * u / N) * odd[u]  # conquer

    return G
\end{python}