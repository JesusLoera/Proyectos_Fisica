\section{Valores esperados y el Principio de Indeterminación de Heisenberg}
\label{Valores esperados y el Principio de Indeterminación de Heisenberg}

En mecánica cuántica, el valor esperado no es más que el promedio de repetidas mediciones en un ensamble de sistemas 
idénticamente preparados. \emph{Obs. No es lo mismo que el promedio de repetidas mediciones en un mismo y 
único sistema}. \\
Dentro del formalismo de la mecánica cuántica, resulta ser que todas las variables dinámicas clásicas pueden ser expresadas 
en términos de la posición y del momento. Si se define el operador posición como:
\begin{equation}
    \hat{x} = x
\end{equation}

Y al operador momento como:
\begin{equation}
    \hat{p} = \frac{\hbar}{i} \frac{\partial}{\partial x}
\end{equation}

Entonces para calcular el valor esperado de cualquier cantidad $Q(x,p)$, simplemente se reemplaza cada variable por su 
operador análogo, se inserta el operador resultante entre $\Psi^*$ y $\Psi$ y finalmente se integra:
\begin{equation}
    \langle Q(x,p) \rangle = \int \Psi^* \hat{Q}(\hat{x},\hat{p}) \Psi \, dx
\end{equation}

Se define además una magnitud conocida como la \emph{desviación estándar} para alguna función arbitraria $j$:
\begin{equation}
    \sigma = \sqrt{\langle j^2 \rangle - \langle j \rangle^2}
\end{equation}

Al lidiar con ondas, se desprende del análisis de Fourier que mientras más precisa sea la posición de una onda, menos 
precisa será será su longitud de onda y viceversa. Como la longitud de onda de una partícula está relacionada con su 
momento \emph{(Véanse, relaciones de De Broglie)}, entonces una dispersión en la longitud de onda corresponde a una 
dispersión en el momento. Lo anterior dicho implica que mientras más precisa sea la posición de una partícula, menos preciso 
será su momento. Cuantitativamente:
\begin{equation}
    \sigma_{x} \sigma_{p} \geq \frac{\hbar}{2}
\end{equation}

Donde $\sigma_{x}$ es la desviación estándar en $x$ y $\sigma_{p}$ es la desviación estándar en $p$. Este es el famoso 
\emph{Principio de Indeterminación de Heisenberg}.\\

Se pretende entonces verificar este principio con el ejemplo anterior del paquete de ondas gaussiano. A partir de la ec. 
$(18)$ se calculan los valores esperados tanto para la posición como para el momento:
\begin{align*}
    \langle x \rangle &= \int_{-\infty}^{\infty} \Psi^* \hat{x} \Psi \, dx = \int_{-\infty}^{\infty} x \lvert\Psi\rvert^2\,dx \\
    &= \int_{-\infty}^{\infty} x \sqrt{\frac{2}{\pi}} \omega e^{-2\omega^2x^2} \,dx
\end{align*}

Y por ser el integrando una función impar:
\begin{equation*}
    \langle x \rangle = 0
\end{equation*}

El valor esperado del momento puede ser obtenido a través de su operador sobre la función de onda, o bien, de una forma 
equivalente y totalmente válida (que además es más rápida):
\begin{equation}
    \langle p \rangle = m \frac{d\langle x \rangle}{dt}
\end{equation}

Y como $\langle x \rangle=0$:
\begin{equation*}
    \langle p \rangle = 0
\end{equation*}

Se calculan a continuación los valores esperados de $x^2$ y de $p^2$, para poder introducirlos en la ecuación $(19)$ de la 
desviación estándar:
\begin{align*}
    \langle x^2 \rangle &= \int_{-\infty}^{\infty} \Psi^* (\hat{x})^2 \Psi \, dx = \int_{-\infty}^{\infty} x^2 \lvert\Psi\rvert^2\,dx \\
    &= \int_{-\infty}^{\infty} x^2 \sqrt{\frac{2}{\pi}} \omega e^{-2\omega^2x^2} \,dx \\
    &= \sqrt{\frac{8}{\pi}} \omega \int_{0}^{\infty} x^2 e^{-2\omega^2x^2} \,dx
\end{align*}

Efectuando cambios de variable apropiados:
\begin{gather*}
    \langle x^2 \rangle = \frac{1}{2\omega^2\sqrt{\pi}} \int_{0}^{\infty} t^{1/2} e^{-t}\,dt 
    = \frac{1}{2\omega^2\sqrt{\pi}} \,\Gamma(\frac{3}{2}) \\
    \langle x^2 \rangle = \frac{1}{4\omega^2}
\end{gather*}

Para $p^2$:
\begin{align*}
    \langle p^2 \rangle &= \int_{-\infty}^{\infty} \Psi^* (\hat{p})^2 \Psi \, dx = \int_{-\infty}^{\infty} \Psi^* \left( \frac{\hbar}{i} \frac{\partial}{\partial x} \right)^2 \Psi \,dx \\
    &= -\hbar^2 \int_{-\infty}^{\infty} \Psi^* \frac{\partial^2\Psi}{\partial x^2}\,dx
\end{align*}

Donde $\Psi(x,t)$ es como se definió en la ec. $(14)$. Esta integral es algo extensa de realizar, así que se proporciona 
el resultado directo, el cual al igual que en los casos anteriores, puede ser calculado mediante las herramientas de la 
librería \emph{Scipy}. Entonces:
\begin{equation*}
    \langle p^2 \rangle = a\hbar^2
\end{equation*}

Se calcula ahora la desviación estándar en $x$ y en $p$:
\begin{align*}
    \sigma_{x} = \sqrt{\langle x^2 \rangle - \langle x \rangle^2} = \sqrt{\frac{1}{4\omega^2}-0} = \frac{1}{2\omega} \\
    \sigma_{p} = \sqrt{\langle p^2 \rangle - \langle p \rangle^2} = \sqrt{a\hbar^2-0} = \hbar\sqrt{a}
\end{align*}

Obteniendo el producto $\sigma_{x}\sigma_{p}$:
\begin{align*}
    \sigma_{x}\sigma_{p} &= \left( \frac{1}{2\omega} \right) \left( \hbar\sqrt{a} \right) = \frac{\hbar}{2} \sqrt{\frac{1+\theta^2}{a}} \sqrt{a} \\
    &= \frac{\hbar}{2} \sqrt{1+\theta^2}
\end{align*}

Recordando la definición de $\theta = \frac{2\hbar a}{m}t$, se tiene que para cualquier tiempo $t$:
\begin{equation*}
    [\theta(t)]^2 \geq 0
\end{equation*}

De aquí se halla entonces una cota inferior para el producto de las desviaciones:
\begin{equation*}
    \sigma_{x}\sigma_{p} = \frac{\hbar}{2} \sqrt{1+\theta^2} \geq \frac{\hbar}{2}
\end{equation*}

Que comparándolo con $(20)$, no es más que una manifestación del Principio de Indeterminación de Heisenberg.