\section{La transformada de Fourier}
\label{La transformada de Fourier}

Las transformadas integrales como lo son la transformada de Laplace o la transformada de Fourier, tienen una
gran aplicación en muchas ramas de la ciencia, en este trabajo nos centraremos en la transformada de Fourier
que puede ser usada para simplificar complejos problemas matemáticos. De manera poco rigurosa podemos decir
que nos lleva una función del espacio en el que está definida al espacio de las frecuencias.


Sea f(x) una función que satisface las condiciones de Dirichlet, entoces se dice que tiene una
representación en series de Fourier.


\begin{theorem}[Teorema de Dirichlet]
  Si se cumple que:
  \begin{enumerate}
    \item Si $f(x)$ tiene periodo $2L$ y es univaluada entre $-L$ y $L$.
    \item Tiene un número finito de máximos y mínimos.
    \item Tiene un número finito de discontinuidades.
    \item Y si $\int_{-L}^{L} \| f(x) \| dx $ es finita.
  \end{enumerate}
  Entonces f(x) tiene representación en series de Fourier y converge a ella en todos los puntos donde
  $f(x)$ es continua y al punto medio donde $f(x)$ es discontinua.
\end{theorem}


\begin{definition}[Series de Fourier]
  Si $f(x)$ satisface las condiciones de Dirichlet entonces su representación en series de Fourier está dada por:

  \begin{equation}
    f(x) = \sum_{n = -\infty}^{\infty} c_{n} e^{i\left( n\pi x / L \right) }
  \end{equation}

\end{definition}


Donde los coeficientes $\left\{ c_{n} \right\}$ suelen ser llamados como el \emph{espectro de f(x)} .


El espectro anterior es un espectro discreto que puede representar ciertas funciones periodicas, sin embargo, existen
otras funciones periodicas o ciertos fenomenos físicos en los que se necesita de un espectro continuo. 
\begin{definition}[La transformada de Fourier]
  Si una función $f(x)$ satisface las condiciones de Dirichlet y si $\int_{-\infty}^{\infty} \| f(x) \| dx $
  es finita entonces:

  \begin{equation}
    F(k) = \frac{1}{\sqrt{2\pi}} \int_{-\infty}^{\infty} f(x) e^{ikx} dx
  \end{equation}

  Y la transformada inversa de Fourier está dada por:

  \begin{equation}
    f(x) = \frac{1}{\sqrt{2\pi}} \int_{-\infty}^{\infty} F(k) e^{-ikx} dk
  \end{equation}

\end{definition}

\section{La transformada de discreta de Fourier DFT}
\label{La transformada de discreta de Fourier DFT}

Notemos que una transformada de Fourier implica la evaluación de integrales y cierto manejo del álgebra, sin embargo,
es posible implementar un algoritmo que nos permita encontrar la transformada de Fourier a un conjunto de datos
discreto que contenga la información de la función a transformar. A este algortimo le llamamos
la \emph{Transformada Discreta de Fourier o DFT}.

\begin{definition}[DFT]
  Sea $X_{n}$ un arreglo con N terminos que contiene la información de una función, la Transformada
  Discreta de Fourier (DFT) de cada elemento del arreglo numérico está por:

  \begin{equation}
    X_{k} = \sum_{n = 1}^{N-1} x_{n} e^{-i2\pi kn/N}
  \end{equation}

  Y la Transformada Inversa Discreta de Fourier (tambíen denotada por DFT inversa o IDFT) está dada por:

  \begin{equation}
    x_{n} = \frac{1}{N}  \sum_{n = 1}^{N-1} X_{k} e^{i2\pi kn/N}
  \end{equation}

\end{definition}

Resaltamos que tanto $x_{n}$ como $X_{k}$ pueden tomar valores complejos.

\section{La transformada rapida de Fourier DFT}
\label{La transformada rapida de Fourier FFT}

Por lo general, tenemos que lidiar con una gran cantidad de puntos de datos, y la velocidad del algoritmo
para la transformada de Fourier se convierte en un tema muy importante. El algoritmo de la DFT tiene un costo
computacional muy grande, conforme crece el tamaño del conjunto de datos, de hecho su complejidad es de
$O(n^{2})$ (donde n es el tamaño del conjunto de datos) .


La Transformada Rápida de Fourier (FFT) es un algoritmo eficiente para calcular la DFT de una secuencia.
Se describe por primera vez en el artículo clásico de Cooley y Tukey en 1965, aunque se hizo referencia
a este por primera vez en un trabajo no publicado de Gauss en 1805. Este algoritmo reduce considerablemente
los costos computacionales pues su complejidad es de $O(nlog(n))$ (donde n es el tamaño del conjunto de datos) .


Este algoritmo se aprovecha de varias simetrías a la hora de calcular la DFT, y será implementado en Python para
el desarrollo de este trabajo.